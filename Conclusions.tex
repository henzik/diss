\section{Conclusions}
Throughout this project, we developed Stamped, a general purpose solution using a smartphone with several advantages over traditional loyalty methods. We completed our objectives outlined in Section \ref{sec:objectives}; moreover, we showed uses for the system in a variety of environments.

In our study we identified that gamification, though somewhat controversial for some users, was an enjoyable motivator for our system. Moreover, we showed that users prefer and see the value of the `tapping' method over classical stamp-cards.  

With the cost of developing mobile applications at an all time high, the barriers of entry for small businesses can be too difficult. With Stamped, we believe that we have developed a flexible novel way for any business, big or small, to create and deploy a simple loyalty schemes. The generality of the system affords many different types of loyalty schemes --- from gym passes to bike miles to coffee shops. Although developed on Android, the protocols are generic allowing the system to be homogeneous to any mobile operating system or even smart card.

With Stamped, we believe that we have contributed a system that gives rise to NFC in our modern lives. Eventhough for a long time, this technology has been seen as a niche, we hope that this inspires a new generation of NFC solutions that solve everyday problems. 

\section{Future Work}
Even though the system has met the aims and objectives we outlined, there are many directions we can expand the project with regards to future work.
\subsection{Making The System Viable}
The system in its current state requires certain additional features to support adoption, some of these include requirements which have yet to be implemented:
\begin{itemize}
	\item Improving the NFC architecture to avoid the problem `weak interactions' 
	\item Encrypting messages between the applications to protect the system from abuse
	\item The ability to manage/customize loyalty-schemes in the Stamped Manager
	\item More options for business to incorporate their brand into their loyalty scheme
\end{itemize}

\subsection{Further Studies}
As well as being able to run a larger study with more participants, it is also of value to study these system modifications
\begin{itemize}
	\item Experimenting with different types of gamification techniques to see which are most effective for our system
	\item Experimenting with suggesting for users (i.e. learning what kinda of loyalty schemes a user frequents)/providing suggestions similar to the way \emph{Amazon} provides suggestions based on user habits
\end{itemize}

\subsection{Extending Stamped}
The functionality of stamped can be extended by the following:
\begin{itemize}
	\item Adding a social element (collecting schemes with friends, sharing rewards)
	\item Allowing for different types of loyalty schemes, not just stamps based
	\item Integrating maps to identify nearby schemes
\end{itemize}

Additionally we can incorporate some of our `Wont Have' requirements we discussed in Chapter 3:
\begin{itemize}
\item The ability to make payments with the system as well as collect stamps
\item Allowing the ability to gift/share rewards with friends (as a metaphor for gift cards) 
\end{itemize}

The application can also be expanded in the following ways:
\begin{itemize}
\item Porting to iOS and Windows platforms
\item Deep integration with Electronic Point Of Sales (EPOS) systems
\item Incorporating further smartcard activities using NFC 
\item Porting to smartwatches (using NFC found in smartwatches to collect stamps as well as the phone)
\item Adding a health element, tracking a users physical activity and rewarding them accordingly
\item Integrating all of the uses of a company/university smartcard (e.g. access to site, payments) as functionality
\end{itemize} 

We believe that Stamped promotes a novel method of managing a daily inconvenience. As a proof of concept, we outlined many directions several different directions the application can evolve to fit into the lives of the user. Perhaps the system could even be revised to incorporate more card-based services such as ticketing.

We're living in a time where contactless technologies are seeing heavy investment --- from bank cards to interactive toys, users are being exposed to newer and more efficient methods to enrich once tedious tasks. Is it possible that one day we'll live in a world where you can interact with anything with a tap? 